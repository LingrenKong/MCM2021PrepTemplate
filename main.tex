% !TEX program = pdflatex
% !TEX encoding = utf8

%%前面那个如果实际操作,完全不含中文可以用pdflatex引擎,注释有utf8没问题,但是正文要小心打出中文符号来。
\documentclass{mcmthesis} %文档类
\usepackage{float} %用于强行定位浮动体,参数H
\mcmsetup{CTeX = false,   % 不使用 CTeX 套装时
        tcn = 0000,  %此为团队码
        problem = A, %问题号
        sheet = true, %
        titleinsheet = true, %
        keywordsinsheet = true,%要关键字
        titlepage = false, %不要这个title
        abstract = true %启用摘要
}

\usepackage{newtxtext} %可以优化数学字体(模板案例也这样来用了)

\title{The \LaTeX{} Template for MCM Version \MCMversion, Preparing!} %标题名称Version \MCMversion



\begin{document} %开始正文

\begin{abstract} %此为摘要环境
    abstract here

\begin{keywords} %关键词
    keyword1; keyword2
\end{keywords}

\end{abstract}
\maketitle 

%(这个文档类仍然需要\maketitle)

\section{Section1}
%%%%%%%%注释
this is sec1 $\beta_1^2 $ 

\begin{thebibliography}{99} %引用部分
    \bibitem{1} D.~E. KNUTH   The \TeX{}book  the American
    Mathematical Society and Addison-Wesley
    Publishing Company , 1984-1986.
    \bibitem{2}Lamport, Leslie,  \LaTeX{}: `` A Document Preparation System '',
    Addison-Wesley Publishing Company, 1986.
    \bibitem{3}\url{https://www.latexstudio.net/}
\end{thebibliography}
    
\begin{appendices} %附录部分
    \section{The prefix(Appendix A) is default}
\end{appendices}

\end{document}